\documentclass{beamer} % imports amsmath symbols

\usepackage{mathleague}

\newcommand{\contestproblemset}{12114}
\newcommand{\roundname}{Team}
\newcommand{\problemnumber}{3}

\begin{document}

\begin{frame} % Title Page
  \titlepage
\end{frame}

\section{Problem}

\subsection*{Identify the objective.}

\begin{frame}
  \frametitle{}
  A whole number value is assigned to every letter in the English alphabet. The \textit{score} of a word is the sum of the values of each letter in the word. Suppose the scores of the words $ERRANT$, $RERUN$, and $AUNT$ are $67$, $49$, and $37$ respectively, and the value of $N$ is the average of the values of $A$ and $T$. What is the value that was assigned to the letter $N$?\pause
  \begin{center}
    Objective: Compute the value assigned to the letter $N$.
  \end{center}
\end{frame}

\section{Solution}

\subsection*{Compute the value assigned to the letter $N$.}

\begin{frame}
  \begin{center}
    Let the value of each letter be denoted by its lowercase.\pause
    
    $a = $ the value of $A$\pause
    
    $b = $ the value of $B$\pause
    
    $\vdots$\pause
  \end{center}

  \begin{equation*}
    2\cdot r+e+a+t+n\pause=67,\pause
  \end{equation*}
  \begin{equation*}
    2\cdot r+e+u+n\pause=49,\pause
  \end{equation*}
  \begin{equation*}
    a+t+u+n\pause=37.
  \end{equation*}
\end{frame}

\begin{frame}
  \begin{align}
    2\cdot r+e+a+t+n &= 67 \\
    2\cdot r+e+u+n &= 49   \\
    a+t+u+n &= 37
  \end{align}\pause
  \[
    (2) - (3) \pause \implies 2\cdot r+e-(a+t)=12
  \]
\end{frame}

\setcounter{equation}{0} % Place this after each frame that uses equations to reset the counter

\begin{frame}
  \begin{align}
    2\cdot r+e+a+t+n &= 67 \\
    2\cdot r+e+u+n &= 49   \\
    a+t+u+n &= 37          \\
    2\cdot r+e-(a+t) &= 12
  \end{align}\pause
  \[
    (1) - (4) \pause \implies 2\cdot(a+t)+n=55
  \]
\end{frame}

\setcounter{equation}{0}

\begin{frame}
  \begin{align}
    2\cdot r+e+a+t+n &= 67 \\
    2\cdot r+e+u+n &= 49   \\
    a+t+u+n &= 37          \\
    2\cdot r+e-(a+t) &= 12 \\
    2\cdot(a+t)+n &= 55
  \end{align}\pause
  Given that the value of $N$ is the average of the values of $A$ and $T$,\pause
  \[
    n=\frac{1}{2}\cdot(a+t) \pause \implies 2 \cdot n = a + t.
  \]\pause
  Substituting $2\cdot n$ for $(a+t)$ in $(5)$, we have that\pause
  \[
    2\cdot 2\cdot n+n=55.
  \]
\end{frame}

\setcounter{equation}{0}

\begin{frame}
  \begin{align}
    2\cdot r+e+a+t+n &= 67 \\
    2\cdot r+e+u+n &= 49   \\
    a+t+u+n &= 37          \\
    2\cdot r+e-(a+t) &= 12 \\
    2\cdot(a+t)+n &= 55
  \end{align}
  Given that the value of $N$ is the average of the values of $A$ and $T$,
  \[
    n=\frac{1}{2}\cdot(a+t) \implies 2 \cdot n = a + t.
  \]
  Substituting $2\cdot n$ for $(a+t)$ in $(5)$, we have that
  \[
    4\cdot n+n=55.
  \]
\end{frame}

\setcounter{equation}{0}

\begin{frame}
  \begin{align}
    2\cdot r+e+a+t+n &= 67 \\
    2\cdot r+e+u+n &= 49   \\
    a+t+u+n &= 37          \\
    2\cdot r+e-(a+t) &= 12 \\
    2\cdot(a+t)+n &= 55
  \end{align}
  Given that the value of $N$ is the average of the values of $A$ and $T$,
  \[
    n=\frac{1}{2}\cdot(a+t) \implies 2 \cdot n = a + t.
  \]
  Substituting $2\cdot n$ for $(a+t)$ in $(5)$, we have that
  \[
    (4+1)\cdot n=55.
  \]
\end{frame}

\setcounter{equation}{0}

\begin{frame}
  \begin{align}
    2\cdot r+e+a+t+n &= 67 \\
    2\cdot r+e+u+n &= 49   \\
    a+t+u+n &= 37          \\
    2\cdot r+e-(a+t) &= 12 \\
    2\cdot(a+t)+n &= 55
  \end{align}
  Given that the value of $N$ is the average of the values of $A$ and $T$,
  \[
    n=\frac{1}{2}\cdot(a+t) \implies 2 \cdot n = a + t.
  \]
  Substituting $2\cdot n$ for $(a+t)$ in $(5)$, we have that
  \[
    5\cdot n = 55.
  \]
\end{frame}

\setcounter{equation}{0}

\begin{frame}
  \begin{align}
    2\cdot r+e+a+t+n &= 67 \\
    2\cdot r+e+u+n &= 49   \\
    a+t+u+n &= 37          \\
    2\cdot r+e-(a+t) &= 12 \\
    2\cdot(a+t)+n &= 55
  \end{align}
  Given that the value of $N$ is the average of the values of $A$ and $T$,
  \[
    n=\frac{1}{2}\cdot(a+t) \implies 2 \cdot n = a + t.
  \]
  Substituting $2\cdot n$ for $(a+t)$ in $(5)$, we have that
  \[
    \frac{1}{5}\cdot 5\cdot n = \frac{1}{5}\cdot 55.
  \]
\end{frame}

\setcounter{equation}{0}

\begin{frame}
  \begin{align}
    2\cdot r+e+a+t+n &= 67 \\
    2\cdot r+e+u+n &= 49   \\
    a+t+u+n &= 37          \\
    2\cdot r+e-(a+t) &= 12 \\
    2\cdot(a+t)+n &= 55
  \end{align}
  Given that the value of $N$ is the average of the values of $A$ and $T$,
  \[
    n=\frac{1}{2}\cdot(a+t) \implies 2 \cdot n = a + t.
  \]
  Substituting $2\cdot n$ for $(a+t)$ in $(5)$, we have that
  \[
    n = \boxed{11}.
  \]
\end{frame}

\setcounter{equation}{0}

\section{Reflection}

\subsection*{Review the concepts.}

\begin{frame}
  \frametitle{Concepts}
  \pause
  \begin{itemize}
    \item elimination \pause
    \item substitution
  \end{itemize}
\end{frame}

%\tcbset{colframe=ml3}
%\begin{frame}
%  \begin{center}
%    \begin{figure}
%      \tcbox{\includegraphics[scale=0.175]{}}
%    \end{figure}
%  \end{center}
%\end{frame}

\end{document}