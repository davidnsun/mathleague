\documentclass{beamer} % imports amsmath symbols

\usepackage{mathleague}

\newcommand{\contestproblemset}{12213}
\newcommand{\roundname}{Team}
\newcommand{\problemnumber}{9}

\begin{document}

\begin{frame} % Title Page
  \titlepage
\end{frame}

\section{Problem}

\subsection*{Identify the objective.}

\begin{frame}
  Let $ a_0 = \frac{1}{4} $ and $ a_n = \frac{1 + a_{n-1}}{2} $ for all $n \ge 1$. Let $ S $ be the sum of the numerator and denominator of $ a_{2021} $ when expressed in simplest form. What is the remainder when $ S $ is divided by 100?
\end{frame}

\section{Solution}

\subsection*{Find the remainder when \texorpdfstring{$S$}{S} is divided by \texorpdfstring{$100$}{100}.}

\begin{frame}
  Trying some small cases, we find that
  \[a_0 = \frac{1}{4}, a_1 = \frac{5}{8}, a_2 = \frac{13}{16}, a_3 = \frac{29}{32}.\]
  We can rewrite these as
  \[a_0 = \frac{2^{0+2}-3}{2^{0+2}}, a_1 = \frac{2^{1+2}-3}{2^{1+2}}, a_2 = \frac{2^{2+2}-3}{2^{2+2}}, a_3 = \frac{2^{3+2}-3}{2^{3+2}}.\]
  \[a_n = \frac{2^{n+2}-3}{2^{n+2}}.\]
\end{frame}

\begin{frame}
  \[a_{2021} = \frac{2^{2023}-3}{2^{2023}}\]
  Since $S$ is the sum of $a_{2021}$'s numerator and denominator, we have
  \[S = 2^{2023} + 2^{2023}-3 = 2^{2024} - 3\]
  \[S \equiv \text{?} \bmod 100\]
\end{frame}

\begin{frame}
  \[S = 2^{2024} - 3\]
  \[S \equiv \text{?} \bmod 100\]
  \only<2->{\[2^{2024} \equiv \text{?} \bmod 100\]}
  \only<3->{\[2^{2024} \equiv \text{?} \bmod 4, \quad 2^{2024} \equiv \text{?} \bmod 25\]}
  \only<4->{\[2^{2024} = (2^2)^{1012} = 4^{1012} \equiv 0 \bmod 4\]}
\end{frame}

\begin{frame}
  \[S = 2^{2024} - 3\]
  \[S \equiv \text{?} \bmod 100\]
  \[2^{2024} \equiv \text{?} \bmod 100\]
  \[2^{2024} \equiv 0 \bmod 4, \quad 2^{2024} \equiv \text{?} \bmod 25\]
  \only<2->{\[2^{10} = 1024 \equiv 24 \equiv -1 \bmod 25\]}
  \only<3->{\[2^{2024} = (2^{10})^{202} \cdot 2^4 \equiv (-1)^{202} \cdot 16 \equiv 16 \bmod 25\]}
\end{frame}

\begin{frame}
  \[S = 2^{2024} - 3\]
  \[S \equiv \text{?} \bmod 100\]
  \[2^{2024} \equiv \text{?} \bmod 100\]
  \[2^{2024} \equiv 0 \bmod 4, \quad 2^{2024} \equiv 16 \bmod 25\]
\end{frame}

\begin{frame}
  \[S = 2^{2024} - 3 \equiv \text{?} \bmod 100\]
  \[2^{2024} \equiv \text{?} \bmod 100\]
  \[2^{2024} \equiv 0 \bmod 4, \quad 2^{2024} \equiv 16 \bmod 25\]
  \only<2->{\[2^{2024} = 16 + 25k, \text{ for some integer } k\]}
  \only<3->{
    For all integers $k$,
    \begin{align*}
      2^{2024} \equiv 0 \bmod 4 &\iff 16 + 25k \equiv 0 \bmod 4\\
      \only<4-6>{&\iff 25k \equiv -16 \bmod 4\\}
      \only<5-6>{&\iff k \equiv 0 \bmod 4\\}
      \only<6->{\implies 16 + 25k \equiv 0 \bmod 4 &\iff k = 0 + 4\ell, \text{ for some integer } \ell }
    \end{align*}
  }
  \only<8->{Thus, $2^{2024} \equiv 0 \bmod 4$ and $2^{2024} \equiv 16 \bmod 25$ if and only if $2^{2024} = 16 + 25 \cdot (0 + 4 \ell) = 16 + 100\ell$, for some integer $\ell$.}
  \only<9->{\[\implies 2^{2024} \equiv 16 \bmod 100\]}
\end{frame}

\begin{frame}
  \[S = 2^{2024} - 3 \equiv \text{?} \bmod 100\]
  \[2^{2024} \equiv 16 \bmod 100\]
  \[2^{2024} \equiv 0 \bmod 4, \quad 2^{2024} \equiv 16 \bmod 25\]
  \[2^{2024} = 16 + 25k, \text{ for some integer } k\]
  For all integers $k$,
  \begin{align*}
    2^{2024} \equiv 0 \bmod 4 &\iff 16 + 25k \equiv 0 \bmod 4\\
    \implies 16 + 25k \equiv 0 \bmod 4 &\iff k = 0 + 4\ell, \text{ for some integer } \ell
  \end{align*}
  Thus, $2^{2024} \equiv 0 \bmod 4$ and $2^{2024} \equiv 16 \bmod 25$ if and only if $2^{2024} = 16 + 25 \cdot (0 + 4 \ell) = 16 + 100\ell$, for some integer $\ell$.
  \[\implies 2^{2024} \equiv 16 \bmod 100\]
\end{frame}

\begin{frame}
  \[S = 2^{2024} - 3 \equiv \boxed{13} \bmod 100\]
  \[2^{2024} \equiv 16 \bmod 100\]
  \[2^{2024} \equiv 0 \bmod 4, \quad 2^{2024} \equiv 16 \bmod 25\]
  \[2^{2024} = 16 + 25k, \text{ for some integer } k\]
  For all integers $k$,
  \begin{align*}
    2^{2024} \equiv 0 \bmod 4 &\iff 16 + 25k \equiv 0 \bmod 4\\
    \implies 16 + 25k \equiv 0 \bmod 4 &\iff k = 0 + 4\ell, \text{ for some integer } \ell
  \end{align*}
  Thus, $2^{2024} \equiv 0 \bmod 4$ and $2^{2024} \equiv 16 \bmod 25$ if and only if $2^{2024} = 16 + 25 \cdot (0 + 4 \ell) = 16 + 100\ell$, for some integer $\ell$.
  \[\implies 2^{2024} \equiv 16 \bmod 100\]
\end{frame}

\setcounter{equation}{0} % Place this after each frame that uses equations to reset the counter

\section{Reflection}

\subsection*{Review the concepts.}

\begin{frame}
  \frametitle{Concepts}
  \begin{itemize}
    \item modular arithmetic
    \item Chinese remainder theorem
  \end{itemize}
\end{frame}

\end{document}