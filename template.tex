\documentclass{beamer} % imports amsmath symbols

\usepackage{mathleague}

\newcommand{\contestproblemset}{}
\newcommand{\roundname}{}
\newcommand{\problemnumber}{}

\begin{document}

\begin{frame} % Title Page
  \titlepage
\end{frame}

\section{Problem}

\subsection*{Identify the objective.}

\begin{frame}
  \begin{tcolorbox}[title=Problem]
    Placeholder text.
  \end{tcolorbox}
\end{frame}

\section{Solution}

\subsection*{Placeholder objective.}

\begin{frame}
  \begin{tcolorbox}[title=Solution]
    \begin{center}
      \begin{tikzpicture}[scale=0.5]
        \draw (0,4.46971821313) node [above left] {$A$};
        \draw (0,0) node [below left] {$B$};
        \draw (14.3185760149,0) node [right] {$C$};
        \draw (0,0) -- (0,4.46971821313);
        \draw (0,0) -- (14.3185760149,0);
        \draw (0,4.46971821313) -- (14.3185760149,0) node [midway,above] {$15$};
        \coordinate (A) at (0,4.46971821313);
        \coordinate (B) at (0,0);
        \coordinate (C) at (14.3185760149,0);
        \tkzMarkRightAngle(A,B,C)
      \end{tikzpicture}
      
      Nice.
    \end{center}
  \end{tcolorbox}
\end{frame}

\setcounter{equation}{0} % Place this after each frame that uses equations to reset the counter

\section{Reflection}

\subsection*{Review the concepts.}

\begin{frame}
  \frametitle{Concepts}
  \begin{itemize}
    \item placeholder 1
    \begin{itemize}
      \item placeholder 2
    \end{itemize}
  \end{itemize}
\end{frame}

\end{document}