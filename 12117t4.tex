\documentclass{beamer} % imports amsmath symbols

\usepackage{mathleague}

\newcommand{\contestproblemset}{12117}
\newcommand{\roundname}{Target}
\newcommand{\problemnumber}{4}

\begin{document}

\begin{frame} % Title Page
  \titlepage
\end{frame}

\section{Problem Statement}

\subsection*{Identify our objective.}

\begin{frame}
An equiangular but not equilateral hexagon has three times the area of a regular hexagon with side length 1. If both hexagons have whole number side lengths, then what is the perimeter of the larger hexagon?
\end{frame}

\section{Solution}

\subsection*{Compute perimeter of the equiangular (but nonregular) hexagon that has thrice the area of a regular hexagon with side lengths $1$.}

\begin{frame}
  \begin{center}
    \begin{tikzpicture}[scale=0.5]
      %\draw (0,4.46971821313) node [above left] {$A$};
      %\draw (0,0) node [below left] {$B$};
      %\draw (14.3185760149,0) node [right] {$C$};
      \draw (0,0) -- (0,4.46971821313);
      \draw (0,0) -- (14.3185760149,0);
      \draw (0,4.46971821313) -- (14.3185760149,0) node [midway,above] {$15$};
      \coordinate (A) at (0,4.46971821313);
      \coordinate (B) at (0,0);
      \coordinate (C) at (14.3185760149,0);
      \tkzMarkRightAngle(A,B,C)
    \end{tikzpicture}
  \end{center}
\end{frame}

\setcounter{equation}{0}

\section{Conclusion}

\subsection*{Review the key concepts we used.}

\begin{frame}
  \frametitle{Key Concepts}
  \pause
  \begin{itemize}
  \item Concept 1 \pause
  \item Concept 2
  \end{itemize}
\end{frame}

\end{document}