\documentclass{beamer} % imports amsmath symbols

\usepackage{mathleague}

\newcommand{\contestproblemset}{12221}
\newcommand{\roundname}{Sprint}
\newcommand{\problemnumber}{25}

\begin{document}

\begin{frame} % Title Page
  \titlepage
\end{frame}

\section{Problem Statement}

\subsection*{Identify our objective.}

\begin{frame}
  On Monday, Travis averaged 45 miles per hour on his roundtrip journey from home to work and then back home. On Tuesday, Travis averaged 40 miles per hour on the same roundtrip journey, and his roundtrip journey took 12 minutes longer than it did on Monday. How far, in miles, is Travis' home from his work?
\end{frame}

\section{Solution}

\subsection*{Compute the distance, in miles, from Travis' home to his work.}

\begin{frame}
Let $d$ be the number of miles from Travis' home to his work. \pause

His roundtrip journey is twice that distance, or $2\cdot d$. \pause

Let $t$ be the number of hours Travis' roundtrip takes on Monday. \pause

We're given that Travis' travel rate on Monday is $45$ mph.\pause

\[
distance = rate \cdot time
\]
\begin{equation}
2\cdot d = 45\cdot t
\end{equation}
\end{frame}

\setcounter{equation}{0}

\begin{frame}
\[
distance = rate \cdot time
\]
\begin{equation}
2\cdot d = 45\cdot t
\end{equation}

\pause On Tuesday, Travis' roundtrip took an additional $12$ minutes, which is $\frac{1}{5}$ of an hour. \pause

The number of hours Travis' roundtrip takes on Tuesday is $t + \frac{1}{5}$. \pause

We're given that Travis' travel rate on Tuesday is $40$ mph.\pause

\begin{equation}
2\cdot d = 40\cdot \left( t + \frac{1}{5} \right)
\end{equation}
\end{frame}

\setcounter{equation}{0}

\begin{frame}
\[
distance = rate \cdot time
\]
\begin{equation}
2\cdot d = 45\cdot t
\end{equation}

On Tuesday, Travis' roundtrip took an additional $12$ minutes, which is $\frac{1}{5}$ of an hour.

The number of hours Travis' roundtrip takes on Tuesday is $t + \frac{1}{5}$.

We're given that Travis' travel rate on Tuesday is $40$ mph.

\begin{equation}
2\cdot d = 40\cdot t + 40\cdot \frac{1}{5}
\end{equation}
\end{frame}

\setcounter{equation}{0}

\begin{frame}
\[
distance = rate \cdot time
\]
\begin{equation}
2\cdot d = 45\cdot t
\end{equation}

On Tuesday, Travis' roundtrip took an additional $12$ minutes, which is $\frac{1}{5}$ of an hour.

The number of hours Travis' roundtrip takes on Tuesday is $t + \frac{1}{5}$.

We're given that Travis' travel rate on Tuesday is $40$ mph.

\begin{equation}
2\cdot d = 40\cdot t + 8
\end{equation}
\end{frame}

\setcounter{equation}{0}

\begin{frame}
\[
distance = rate \cdot time
\]
\begin{equation}
2\cdot d = 45\cdot t
\end{equation}
\begin{equation}
2\cdot d = 40\cdot t + 8
\end{equation}

\pause Substituting $45\cdot t$ for $2\cdot d$ in (2), we have \pause

\[
45\cdot t = 40\cdot t + 8.
\]
\end{frame}

\setcounter{equation}{0}

\begin{frame}
\[
distance = rate \cdot time
\]
\begin{equation}
2\cdot d = 45\cdot t
\end{equation}
\begin{equation}
2\cdot d = 40\cdot t + 8
\end{equation}

Substituting $45\cdot t$ for $2\cdot d$ in (2), we have

\[
45\cdot t - 40\cdot t = 8.
\]
\end{frame}

\setcounter{equation}{0}

\begin{frame}
\[
distance = rate \cdot time
\]
\begin{equation}
2\cdot d = 45\cdot t
\end{equation}
\begin{equation}
2\cdot d = 40\cdot t + 8
\end{equation}

Substituting $45\cdot t$ for $2\cdot d$ in (2), we have

\[
(45 - 40)\cdot t = 8.
\]
\end{frame}

\setcounter{equation}{0}

\begin{frame}
\[
distance = rate \cdot time
\]
\begin{equation}
2\cdot d = 45\cdot t
\end{equation}
\begin{equation}
2\cdot d = 40\cdot t + 8
\end{equation}

Substituting $45\cdot t$ for $2\cdot d$ in (2), we have

\[
5\cdot t = 8.
\]
\end{frame}

\setcounter{equation}{0}

\begin{frame}
\[
distance = rate \cdot time
\]
\begin{equation}
2\cdot d = 45\cdot t
\end{equation}
\begin{equation}
2\cdot d = 40\cdot t + 8
\end{equation}

Substituting $45\cdot t$ for $2\cdot d$ in (2), we have

\[
t = \frac{8}{5}.
\]
\end{frame}

\setcounter{equation}{0}

\begin{frame}
\[
distance = rate \cdot time
\]
\begin{equation}
2\cdot d = 45\cdot t
\end{equation}
\begin{equation}
2\cdot d = 40\cdot t + 8
\end{equation}
\begin{equation}
t = \frac{8}{5}
\end{equation}

\pause Substituting $\frac{8}{5}$ for $t$ in (1), we have \pause

\[
2\cdot d = 45\cdot \frac{8}{5}.
\]
\end{frame}

\setcounter{equation}{0}

\begin{frame}
\[
distance = rate \cdot time
\]
\begin{equation}
2\cdot d = 45\cdot t
\end{equation}
\begin{equation}
2\cdot d = 40\cdot t + 8
\end{equation}
\begin{equation}
t = \frac{8}{5}
\end{equation}

Substituting $\frac{8}{5}$ for $t$ in (1), we have

\[
2\cdot d = 45\cdot \frac{1}{5}\cdot 8.
\]
\end{frame}

\setcounter{equation}{0}

\begin{frame}
\[
distance = rate \cdot time
\]
\begin{equation}
2\cdot d = 45\cdot t
\end{equation}
\begin{equation}
2\cdot d = 40\cdot t + 8
\end{equation}
\begin{equation}
t = \frac{8}{5}
\end{equation}

Substituting $\frac{8}{5}$ for $t$ in (1), we have

\[
2\cdot d = 9\cdot 8.
\]
\end{frame}

\setcounter{equation}{0}

\begin{frame}
\[
distance = rate \cdot time
\]
\begin{equation}
2\cdot d = 45\cdot t
\end{equation}
\begin{equation}
2\cdot d = 40\cdot t + 8
\end{equation}
\begin{equation}
t = \frac{8}{5}
\end{equation}

Substituting $\frac{8}{5}$ for $t$ in (1), we have

\[
2\cdot d = 72.
\]
\end{frame}

\setcounter{equation}{0}

\begin{frame}
\[
distance = rate \cdot time
\]
\begin{equation}
2\cdot d = 45\cdot t
\end{equation}
\begin{equation}
2\cdot d = 40\cdot t + 8
\end{equation}
\begin{equation}
t = \frac{8}{5}
\end{equation}

Substituting $\frac{8}{5}$ for $t$ in (1), we have

\[
d = \boxed{36}.
\]
\end{frame}

\section{Conclusion}

\subsection*{Review the key concepts we used.}

\begin{frame}
  \frametitle{Key Concepts}
  \pause
  \begin{itemize}
  \item Rates \pause
  \item System of equations \pause
  \item Substitution
  \end{itemize}
\end{frame}

\end{document}
